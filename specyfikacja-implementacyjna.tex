\documentclass[10pt,a4paper]{article}
\usepackage[a4paper]{geometry}

\usepackage{polski}
\usepackage{indentfirst}
\usepackage{xltxtra}
\usepackage{relsize}
\usepackage{fancyvrb}
\usepackage[pdfborder={0 0 0}]{hyperref}
\usepackage{graphicx}
\usepackage{changepage}

\defaultfontfeatures{Mapping=tex-text}
\setromanfont{Charis SIL}
\setmonofont[Scale=MatchLowercase]{Menlo}
\linespread{1.25}

\DefineVerbatimEnvironment%
  {SmallVerbatim}%
  {Verbatim}{fontsize=\relsize{-0.5},numbers=left,numbersep=-10pt,frame=lines,tabsize=4}

\newcommand{\f}[1]{\texttt{#1}}

\begin{document}

%%fakesection{Tytuł}
\title{ 
  Interpolacja funkcjami sklejanymi\\
  {\normalsize Specyfikacja implementacyjna projektu nr 2}\\\vspace{-12pt}
  {\normalsize z przedmiotu \emph{Języki i metody programowania 2}}
}
\author{
  Tomasz Cudziło\\
  {\small EE PW, 211552}
}
\date{\today}
\maketitle

\section*{Zadanie}
\label{sec:zadanie}

Napisać program, z~graficznym interfejsem użytkownika, wyznaczający
współczynniki funkcji sklejanych trzeciego stopnia aproksymujących zadany ciąg
danych pomiarowych.

\vspace{20pt}

\section{Logika klas}

Cały projekt jest luźno oparty na wzorcu \f{MVC}. Punktem wejścia jest klasa
\f{SplinesApp}, która posiada obiekt klasy \f{SplinesController}. Ta
instancja kontrolera, posiada obiekty punktów, wielomianów i~wykres, oraz
zarządza nimi zgodnie z~poleceniami użytkownika płynącymi z~instancji
\f{SplinesView}.

Schemat z~rys.~\ref{fig:aplikacja} na stronie \pageref{fig:aplikacja}
przedstawia zarys podziału problemu na klasy.

\newpage
\begin{figure}[ht]
  \begin{adjustwidth}{-3cm}{-3cm}
    \centering
    \includegraphics{figury/aplikacja}
    \caption{Schemat całej aplikacji.}
    \label{fig:aplikacja}
  \end{adjustwidth}
\end{figure}
\clearpage

\subsection{Aplikacja}

Klasa \f{SplinesApp} (rys. \ref{fig:aplikacja-szczegolowo}) jest punktem
wejściowym programu. Jednocześnie jest zaimplementowana jako \f{Singleton},
który służy do wywoływania binarki z~pierwszego projektu.

Metoda \f{main()} inicjuje stan \f{SplinesApp}, to znaczy sprawdza czy
\f{splines} jest widoczne z~\f{PATH}, i~zapamiętuje ścieżkę. Następnie tworzy
i~uruchamia \f{SplinesController}.

Klasa udostępnia metody zwracające dane wygenerowane przez binarkę z~pierwszego
projektu.

\begin{figure}[hb]
  \centering
  \includegraphics{figury/aplikacja-szczegolowo}
  \caption{Schemat \f{SplinesApp}.}
  \label{fig:aplikacja-szczegolowo}
\end{figure}

\subsection{Modele}

\subsection{Kontroler}

\subsection{Widok}

\section{Pliki}

\end{document}
